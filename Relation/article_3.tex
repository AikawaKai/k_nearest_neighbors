%%%%%%%%%%%%%%%%%%%%%%%%%%%%%%%%%%%%%%%%%
% Stylish Article
% LaTeX Template
% Version 2.1 (1/10/15)
%
% This template has been downloaded from:
% http://www.LaTeXTemplates.com
%
% Original author:
% Mathias Legrand (legrand.mathias@gmail.com) 
% With extensive modifications by:
% Vel (vel@latextemplates.com)
%
% License:
% CC BY-NC-SA 3.0 (http://creativecommons.org/licenses/by-nc-sa/3.0/)
%
%%%%%%%%%%%%%%%%%%%%%%%%%%%%%%%%%%%%%%%%%

%----------------------------------------------------------------------------------------
%	PACKAGES AND OTHER DOCUMENT CONFIGURATIONS
%----------------------------------------------------------------------------------------

\documentclass[fleqn,10pt]{SelfArx} % Document font size and equations flushed left

\usepackage{lipsum} % Required to insert dummy text. To be removed otherwise

%----------------------------------------------------------------------------------------
%	COLUMNS
%----------------------------------------------------------------------------------------

\setlength{\columnsep}{0.55cm} % Distance between the two columns of text
\setlength{\fboxrule}{0.75pt} % Width of the border around the abstract

%----------------------------------------------------------------------------------------
%	COLORS
%----------------------------------------------------------------------------------------

\definecolor{color1}{RGB}{0,0,90} % Color of the article title and sections
\definecolor{color2}{RGB}{0,20,20} % Color of the boxes behind the abstract and headings

%----------------------------------------------------------------------------------------
%	HYPERLINKS
%----------------------------------------------------------------------------------------

\usepackage{hyperref} % Required for hyperlinks
\hypersetup{hidelinks,colorlinks,breaklinks=true,urlcolor=color2,citecolor=color1,linkcolor=color1,bookmarksopen=false,pdftitle={Title},pdfauthor={Author}}

%----------------------------------------------------------------------------------------
%	ARTICLE INFORMATION
%----------------------------------------------------------------------------------------

\JournalInfo{Journal, Vol. XXI, No. 1, 1-5, 2013} % Journal information
\Archive{Additional note} % Additional notes (e.g. copyright, DOI, review/research article)


\PaperTitle{\center Progetto di Metodi statistici per l'apprendimento: K-Nearest Neighbors} % Article title
\Authors{\center \textbf{Studente:} Marco Odore - \textbf{Matricola:} 868906}
 % Authors

\Keywords{Keyword1 --- Keyword2 --- Keyword3} % Keywords - if you don't want any simply remove all the text between the curly brackets
\newcommand{\keywordname}{Keywords} % Defines the keywords heading name

%----------------------------------------------------------------------------------------
%	ABSTRACT
%----------------------------------------------------------------------------------------

\Abstract{Implementzione in R e C++ dell'algoritmo K-nearest neighbors e sua applicazione ad un problema di classificazione binaria, nella sua versione classica e online.}

%----------------------------------------------------------------------------------------

\begin{document}

\flushbottom % Makes all text pages the same height

\maketitle % Print the title and abstract box

\tableofcontents % Print the contents section

\thispagestyle{empty} % Removes page numbering from the first page

%----------------------------------------------------------------------------------------
%	ARTICLE CONTENTS
%----------------------------------------------------------------------------------------

\section*{Introduction} % The \section*{} command stops section numbering

\addcontentsline{toc}{section}{Introduction} % Adds this section to the table of contents

Si è scelto di implementare in R l'algoritmo K-NN e di applicarlo ad un problema di classificazione binario, dove si vuole associare ad una persona il suo possibile reddito annuale, date alcune sue informazioni di base. 

\subsection{Il dataset}
Il dataset utilizzato \cite{Lichman:2013} contiene 48842 istanze, ognuna delle quali è caratterizzata da 14 feature:

\begin{itemize}
\item Età - Tipo continuo
\item Workclass - Tipo nominale (8 valori possibili)
\item Fnlwgt - Tipo continuo
\item Education - Tipo nominale (16 valori possibili)
\item Education-num - Tipo continuo (trasformazione di education in tipo continuo)

\end{itemize}

%------------------------------------------------

\section{Methods}

\begin{figure*}[ht]\centering % Using \begin{figure*} makes the figure take up the entire width of the page
\includegraphics[width=\linewidth]{view}
\caption{Wide Picture}
\label{fig:view}
\end{figure*}

\lipsum[4] % Dummy text

\begin{equation}
\cos^3 \theta =\frac{1}{4}\cos\theta+\frac{3}{4}\cos 3\theta
\label{eq:refname2}
\end{equation}

\lipsum[5] % Dummy text

\begin{enumerate}[noitemsep] % [noitemsep] removes whitespace between the items for a compact look
\item First item in a list
\item Second item in a list
\item Third item in a list
\end{enumerate}

\subsection{Subsection}

\lipsum[6] % Dummy text

\paragraph{Paragraph} \lipsum[7] % Dummy text
\paragraph{Paragraph} \lipsum[8] % Dummy text

\subsection{Subsection}

\lipsum[9] % Dummy text

\begin{figure}[ht]\centering
\includegraphics[width=\linewidth]{results}
\caption{In-text Picture}
\label{fig:results}
\end{figure}

Reference to Figure \ref{fig:results}.

%------------------------------------------------

\section{Results and Discussion}

\lipsum[10] % Dummy text

\subsection{Subsection}

\lipsum[11] % Dummy text

\begin{table}[hbt]
\caption{Table of Grades}
\centering
\begin{tabular}{llr}
\toprule
\multicolumn{2}{c}{Name} \\
\cmidrule(r){1-2}
First name & Last Name & Grade \\
\midrule
John & Doe & $7.5$ \\
Richard & Miles & $2$ \\
\bottomrule
\end{tabular}
\label{tab:label}
\end{table}

\subsubsection{Subsubsection}

\lipsum[12] % Dummy text

\begin{description}
\item[Word] Definition
\item[Concept] Explanation
\item[Idea] Text
\end{description}

\subsubsection{Subsubsection}

\lipsum[13] % Dummy text

\begin{itemize}[noitemsep] % [noitemsep] removes whitespace between the items for a compact look
\item First item in a list
\item Second item in a list
\item Third item in a list
\end{itemize}

\subsubsection{Subsubsection}

\lipsum[14] % Dummy text

\subsection{Subsection}

\lipsum[15-23] % Dummy text

%------------------------------------------------
\phantomsection
\section*{Acknowledgments} % The \section*{} command stops section numbering

\addcontentsline{toc}{section}{Acknowledgments} % Adds this section to the table of contents


%----------------------------------------------------------------------------------------
%	REFERENCE LIST
%----------------------------------------------------------------------------------------
\phantomsection
\bibliography{article_3}
\bibliographystyle{unsrt}



%----------------------------------------------------------------------------------------

\end{document}